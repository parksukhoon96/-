%%%%%%%%%%%%%%%%%%%% author.tex %%%%%%%%%%%%%%%%%%%%%%%%%%%%%%%%%%%
%
% sample root file for your "contribution" to a contributed volume
%
% Use this file as a template for your own input.
%
%%%%%%%%%%%%%%%% Springer %%%%%%%%%%%%%%%%%%%%%%%%%%%%%%%%%%


% RECOMMENDED %%%%%%%%%%%%%%%%%%%%%%%%%%%%%%%%%%%%%%%%%%%%%%%%%%%
\documentclass[graybox]{svmult}

% choose options for [] as required from the list
% in the Reference Guide

\usepackage{type1cm}        % activate if the above 3 fonts are
                            % not available on your system
\usepackage{kotex}                            
%
\usepackage{makeidx}         % allows index generation
\usepackage{graphicx}        % standard LaTeX graphics tool
                             % when including figure files
\usepackage{multicol}        % used for the two-column index
\usepackage[bottom]{footmisc}% places footnotes at page bottom


\usepackage{newtxtext}       % 
\usepackage{newtxmath}       % selects Times Roman as basic font

% see the list of further useful packages
% in the Reference Guide

\makeindex             % used for the subject index
                       % please use the style svind.ist with
                       % your makeindex program

%%%%%%%%%%%%%%%%%%%%%%%%%%%%%%%%%%%%%%%%%%%%%%%%%%%%%%%%%%%%%%%%%%%%%%%%%%%%%%%%%%%%%%%%%

\begin{document}

\title*{사회발전이란 무엇인가?}
% Use \titlerunning{Short Title} for an abbreviated version of
% your contribution title if the original one is too long
\author{박석훈}
% Use \authorrunning{Short Title} for an abbreviated version of
% your contribution title if the original one is too long
\institute{\at 박석훈, 서울대학교 사회학과, \email{parksukhoon@snu.ac.kr}
}
%
% Use the package "url.sty" to avoid
% problems with special characters
% used in your e-mail or web address
%
\maketitle

\section{세 영역의 조화로운 가능성은 어디에 있는가?}
세르주 라투슈(Serge Latouche)의 지적 이전부터, 경제 발전만을 강조함으로써 사회와 환경 부문의 중요성이 간과되고 있다는 주장은 지속되어왔다. 국가, 사회, 개인 세 주체를 동시에 고려하면서 경제와 사회 환경을 함께 다루고 갈등을 효과적으로 다루는 것은 현실적으로 가능한 일인가?  작년 지급된 전국민 1차 재난지원금은 약 14조 원이었으며, 세제 혜택을 보장해주었음에도 `자발적 기부금'은 287억원으로 전체 지원금의 0.2\%이다. 합리적 선택 이론(Rational Choice Theory)에 따르면 이러한 현상은 너무도 당연한 것이다. 미래세대를 현재세대의 합리성에 포함될 수 있는가? 포괄적인 `발전' 개념은 사회적 합의가 필요하다. 여러 이해관계자가 동의할 수 있는 발전은 가능한가? 사회적 합의를 위한 선결조건에는 어떠한 것들이 있을까?

\section{`사회발전=경제발전'은 부당한 명제인가?}
  국민의식조사 결과 개인의 건강, 가족과의 관계를 제외하고 돈을 많이 버는 것과 좋은 일자리를 갖는 것이 국민들 사이에서 `최우선으로 중요'하다고 응답한 비율이 가장 높았으며, `중요하지만 최우선순위는 아님'이라고 응답한 비율 역시 각각 67.6\%와 71.7\%에 달한다. 또한, 사회발전을 위한 가치의 우선순위를 묻는 문항에 1순위 응답으로 `안정적인 경제성장'이 22\%로 가장 높게 나타났다. `국가사회 발전지수 개발을 위한 이론체계 연구' 보고서에 나타난 전문가 설문조사 결과 역시 사회발전에 대한 인식은 경제발전에 치중되어있으며, 환경이나 국제협력 분야에는 낮은 가치를 부여하고 있으며, 환경은 당장의 가치와는 거리가 먼, `미래세대'의 문제로 나타났다. 경제발전의 신화는 한국사회에 여전히 존재한다. \\
 
  한국 사회에서 경제발전과 사회발전의 구분점은 여전히 모호(ambiguous)하며, 더 큰 경제성장을 통해 문제를 해결할 수 있다는 인식이 만연해 보인다. 부르디외의 자본 개념은 경제자본에 근거한다(Bourdieu, 1986). 꿈-자본이 ``꿈을 꾸고 이를 유지하며 그것의 실현을 향해 나가는 능력의 총체"라면, 꿈-자본 역시 경제자본과 무관할 수 없다. 한국 사회 청년들이 제기하는 공정성 담론에서 `경제자본'을 제외할 수 있는가? 더 극단적으로, `경제자본'과 무관한 영역이 있을 수 있는가?\\
 
 세르주 라트슈는 `발전에서 살아남기'에서 ``우리가 자발적으로 발전을, 우리의 생활 방식을, 그것들에 연결된 기술을 거부하지 않으리라는 점은 분명하다"라고 서술하였다. 현재의 발전 모델을 포기하지 않는 현대인들에게 발전의 의미를 다시 묻는 작업은 어떤 의미를 지닐 수 있는가? 혹은 경제위기는 경제성장이 아닌 사회와 환경 (혹은 국제) 영역에서 해결할 수 있는가? 위 문장들은 모두 경제발전과 구분되는 사회발전만의 지점이 어디에 있는지를 묻는 질문과 같다.
 
\section{이질성 포착은 세대 내 갈등 완화에 도움을 줄 수 있는가?}
 정치적 기획 혹은 기성세대 관점에서 비롯된 세대 담론이 `청년'이라는 집단을 동질적인 집단으로 여기고 있다. 즉, 청년이라고 해서 다 같은 청년이 아니며, 자본-꿈-자본을 포함한-의 양이 상이하다. 예를 들어, 청년여성, 청년남성, 중년여성, 중년남성은 `기회가 되면 내 것을 타인에게 나누어줄 수 있다'는 명제에 대해 이질적인 태도를 보인다. 하지만, 성별, 연령별, 계층별 등 다양한 차원에서 이질성을 포착하고 취약성(precarity)을 찾고 지원해주는 행위는 이질적인 정책을 수반한다. `청년적금통장'에 가장 많은 공감을 받은 댓글은 4050 세대에도 그러한 기회를 달라는 댓글이었다. 나와 성별이 다르다는 이유로, 혹은 나보다 몇 년 더 늦게 태어났다는 이유로 누군가가 배타적 기회를 얻는다면 세대 내 갈등은 완화될 수 있는가? `세대 내 갈등'은 취약점을 보완하는 방식으로 해결될 수 있는가? 더 나아가, 우리 사회는 그러한 취약점을 인정하고 수용할 수 있는 기반이 갖추어져 있는가? 그러한 기반에는 무엇이 있을까?

%\input{references}
\end{document}
