%%%%%%%%%%%%%%%%%%%% author.tex %%%%%%%%%%%%%%%%%%%%%%%%%%%%%%%%%%%
%
% sample root file for your "contribution" to a contributed volume
%
% Use this file as a template for your own input.
%
%%%%%%%%%%%%%%%% Springer %%%%%%%%%%%%%%%%%%%%%%%%%%%%%%%%%%


% RECOMMENDED %%%%%%%%%%%%%%%%%%%%%%%%%%%%%%%%%%%%%%%%%%%%%%%%%%%
\documentclass[graybox]{svmult}

% choose options for [] as required from the list
% in the Reference Guide

\usepackage{type1cm}        % activate if the above 3 fonts are
                            % not available on your system
\usepackage{kotex}                            
%
\usepackage{makeidx}         % allows index generation
\usepackage{graphicx}        % standard LaTeX graphics tool
                             % when including figure files
\usepackage{multicol}        % used for the two-column index
\usepackage[bottom]{footmisc}% places footnotes at page bottom


\usepackage{newtxtext}       % 
\usepackage{newtxmath}       % selects Times Roman as basic font

% see the list of further useful packages
% in the Reference Guide

\makeindex             % used for the subject index
                       % please use the style svind.ist with
                       % your makeindex program

%%%%%%%%%%%%%%%%%%%%%%%%%%%%%%%%%%%%%%%%%%%%%%%%%%%%%%%%%%%%%%%%%%%%%%%%%%%%%%%%%%%%%%%%%

\begin{document}

\title*{사회발전이란 무엇인가?}
% Use \titlerunning{Short Title} for an abbreviated version of
% your contribution title if the original one is too long
\author{박석훈}
% Use \authorrunning{Short Title} for an abbreviated version of
% your contribution title if the original one is too long
\institute{\at 박석훈, 서울대학교 사회학과, \email{parksukhoon@snu.ac.kr}
}
%
% Use the package "url.sty" to avoid
% problems with special characters
% used in your e-mail or web address
%
\maketitle

\section{세 영역의 조화로운 가능성은 어디에 있는가?}
세르주 라투슈(Serge Latouche)의 지적 이전부터, 경제 발전만을 강조함으로써 사회와 환경 부문의 중요성이 간과되고 있다는 주장은 지속되어왔다. 국가, 사회, 개인 세 주체를 동시에 고려하면서 경제와 사회 환경을 함께 다루고 갈등을 효과적으로 다루는 것은 현실적으로 가능한 일인가?  작년 지급된 전국민 1차 재난지원금은 약 14조 원이었으며, 세제 혜택을 보장해주었음에도 `자발적 기부금'은 287억원으로 전체 지원금의 0.2\%이다. 합리적 선택 이론(Rational Choice Theory)에 따르면 이러한 현상은 너무도 당연한 것이다. 미래세대를 현재세대의 합리성에 포함될 수 있는가? 포괄적인 `발전' 개념은 사회적 합의가 필요하다. 여러 이해관계자가 동의할 수 있는 발전은 가능한가? 사회적 합의를 위한 선결조건에는 어떠한 것들이 있을까?

\section{`사회발전=경제발전'은 부당한 명제인가?}
  국민의식조사 결과 개인의 건강, 가족과의 관계를 제외하고 돈을 많이 버는 것과 좋은 일자리를 갖는 것이 국민들 사이에서 `최우선으로 중요'하다고 응답한 비율이 가장 높았으며, `중요하지만 최우선순위는 아님'이라고 응답한 비율 역시 각각 67.6\%와 71.7\%에 달한다. 또한, 사회발전을 위한 가치의 우선순위를 묻는 문항에 1순위 응답으로 `안정적인 경제성장'이 22\%로 가장 높게 나타났다. `국가사회 발전지수 개발을 위한 이론체계 연구' 보고서에 나타난 전문가 설문조사 결과 역시 사회발전에 대한 인식은 경제발전에 치중되어있으며, 환경이나 국제협력 분야에는 낮은 가치를 부여하고 있으며, 환경은 당장의 가치와는 거리가 먼, `미래세대'의 문제로 나타났다. 경제발전의 신화는 한국사회에 여전히 존재한다. \\
 
  한국 사회에서 경제발전과 사회발전의 구분점은 여전히 모호(ambiguous)하며, 더 큰 경제성장을 통해 문제를 해결할 수 있다는 인식이 만연해 보인다. 부르디외의 자본 개념은 경제자본에 근거한다(Bourdieu, 1986). 꿈-자본이 ``꿈을 꾸고 이를 유지하며 그것의 실현을 향해 나가는 능력의 총체"라면, 꿈-자본 역시 경제자본과 무관할 수 없다. 한국 사회 청년들이 제기하는 공정성 담론에서 `경제자본'을 제외할 수 있는가? 더 극단적으로, `경제자본'과 무관한 영역이 있을 수 있는가?\\
 
 세르주 라트슈는 `발전에서 살아남기'에서 ``우리가 자발적으로 발전을, 우리의 생활 방식을, 그것들에 연결된 기술을 거부하지 않으리라는 점은 분명하다"라고 서술하였다. 현재의 발전 모델을 포기하지 않는 현대인들에게 발전의 의미를 다시 묻는 작업은 어떤 의미를 지닐 수 있는가? 혹은 경제위기는 경제성장이 아닌 사회와 환경 (혹은 국제) 영역에서 해결할 수 있는가? 위 문장들은 모두 경제발전과 구분되는 사회발전만의 지점이 어디에 있는지를 묻는 질문과 같다.
 
\section{이질성 포착은 세대 내 갈등 완화에 도움을 줄 수 있는가?}
 정치적 기획 혹은 기성세대 관점에서 비롯된 세대 담론이 `청년'이라는 집단을 동질적인 집단으로 여기고 있다. 즉, 청년이라고 해서 다 같은 청년이 아니며, 자본-꿈-자본을 포함한-의 양이 상이하다. 예를 들어, 청년여성, 청년남성, 중년여성, 중년남성은 `기회가 되면 내 것을 타인에게 나누어줄 수 있다'는 명제에 대해 이질적인 태도를 보인다. 하지만, 성별, 연령별, 계층별 등 다양한 차원에서 이질성을 포착하고 취약성(precarity)을 찾고 지원해주는 행위는 이질적인 정책을 수반한다. `청년적금통장'에 가장 많은 공감을 받은 댓글은 4050 세대에도 그러한 기회를 달라는 댓글이었다. 나와 성별이 다르다는 이유로, 혹은 나보다 몇 년 더 늦게 태어났다는 이유로 누군가가 배타적 기회를 얻는다면 세대 내 갈등은 완화될 수 있는가? `세대 내 갈등'은 취약점을 보완하는 방식으로 해결될 수 있는가? 더 나아가, 우리 사회는 그러한 취약점을 인정하고 수용할 수 있는 기반이 갖추어져 있는가? 그러한 기반에는 무엇이 있을까?

%%%%%%%%%%%%%%%%%%%%%%%%% referenc.tex %%%%%%%%%%%%%%%%%%%%%%%%%%%%%%
% sample references
% %
% Use this file as a template for your own input.
%
%%%%%%%%%%%%%%%%%%%%%%%% Springer-Verlag %%%%%%%%%%%%%%%%%%%%%%%%%%
%
% BibTeX users please use
% \bibliographystyle{}
% \bibliography{}
%
\biblstarthook{References may be \textit{cited} in the text either by number (preferred) or by author/year.\footnote{Make sure that all references from the list are cited in the text. Those not cited should be moved to a separate \textit{Further Reading} section or chapter.} If the citatiion in the text is numbered, the reference list should be arranged in ascending order. If the citation in the text is author/year, the reference list should be \textit{sorted} alphabetically and if there are several works by the same author, the following order should be used:
\begin{enumerate}
\item all works by the author alone, ordered chronologically by year of publication
\item all works by the author with a coauthor, ordered alphabetically by coauthor
\item all works by the author with several coauthors, ordered chronologically by year of publication.
\end{enumerate}
The \textit{styling} of references\footnote{Always use the standard abbreviation of a journal's name according to the ISSN \textit{List of Title Word Abbreviations}, see \url{http://www.issn.org/en/node/344}} depends on the subject of your book:
\begin{itemize}
\item The \textit{two} recommended styles for references in books on \textit{mathematical, physical, statistical and computer sciences} are depicted in ~\cite{science-contrib, science-online, science-mono, science-journal, science-DOI} and ~\cite{phys-online, phys-mono, phys-journal, phys-DOI, phys-contrib}.
\item Examples of the most commonly used reference style in books on \textit{Psychology, Social Sciences} are~\cite{psysoc-mono, psysoc-online,psysoc-journal, psysoc-contrib, psysoc-DOI}.
\item Examples for references in books on \textit{Humanities, Linguistics, Philosophy} are~\cite{humlinphil-journal, humlinphil-contrib, humlinphil-mono, humlinphil-online, humlinphil-DOI}.
\item Examples of the basic Springer Nature style used in publications on a wide range of subjects such as \textit{Computer Science, Economics, Engineering, Geosciences, Life Sciences, Medicine, Biomedicine} are ~\cite{basic-contrib, basic-online, basic-journal, basic-DOI, basic-mono}. 
\end{itemize}
}

\begin{thebibliography}{99.}%
% and use \bibitem to create references.
%
% Use the following syntax and markup for your references if 
% the subject of your book is from the field 
% "Mathematics, Physics, Statistics, Computer Science"
%
% Contribution 
\bibitem{science-contrib} Broy, M.: Software engineering --- from auxiliary to key technologies. In: Broy, M., Dener, E. (eds.) Software Pioneers, pp. 10-13. Springer, Heidelberg (2002)
%
% Online Document
\bibitem{science-online} Dod, J.: Effective substances. In: The Dictionary of Substances and Their Effects. Royal Society of Chemistry (1999) Available via DIALOG. \\
\url{http://www.rsc.org/dose/title of subordinate document. Cited 15 Jan 1999}
%
% Monograph
\bibitem{science-mono} Geddes, K.O., Czapor, S.R., Labahn, G.: Algorithms for Computer Algebra. Kluwer, Boston (1992) 
%
% Journal article
\bibitem{science-journal} Hamburger, C.: Quasimonotonicity, regularity and duality for nonlinear systems of partial differential equations. Ann. Mat. Pura. Appl. \textbf{169}, 321--354 (1995)
%
% Journal article by DOI
\bibitem{science-DOI} Slifka, M.K., Whitton, J.L.: Clinical implications of dysregulated cytokine production. J. Mol. Med. (2000) doi: 10.1007/s001090000086 
%
\bigskip

% Use the following (APS) syntax and markup for your references if 
% the subject of your book is from the field 
% "Mathematics, Physics, Statistics, Computer Science"
%
% Online Document
\bibitem{phys-online} J. Dod, in \textit{The Dictionary of Substances and Their Effects}, Royal Society of Chemistry. (Available via DIALOG, 1999), 
\url{http://www.rsc.org/dose/title of subordinate document. Cited 15 Jan 1999}
%
% Monograph
\bibitem{phys-mono} H. Ibach, H. L\"uth, \textit{Solid-State Physics}, 2nd edn. (Springer, New York, 1996), pp. 45-56 
%
% Journal article
\bibitem{phys-journal} S. Preuss, A. Demchuk Jr., M. Stuke, Appl. Phys. A \textbf{61}
%
% Journal article by DOI
\bibitem{phys-DOI} M.K. Slifka, J.L. Whitton, J. Mol. Med., doi: 10.1007/s001090000086
%
% Contribution 
\bibitem{phys-contrib} S.E. Smith, in \textit{Neuromuscular Junction}, ed. by E. Zaimis. Handbook of Experimental Pharmacology, vol 42 (Springer, Heidelberg, 1976), p. 593
%
\bigskip
%
% Use the following syntax and markup for your references if 
% the subject of your book is from the field 
% "Psychology, Social Sciences"
%
%
% Monograph
\bibitem{psysoc-mono} Calfee, R.~C., \& Valencia, R.~R. (1991). \textit{APA guide to preparing manuscripts for journal publication.} Washington, DC: American Psychological Association.
%
% Online Document
\bibitem{psysoc-online} Dod, J. (1999). Effective substances. In: The dictionary of substances and their effects. Royal Society of Chemistry. Available via DIALOG. \\
\url{http://www.rsc.org/dose/Effective substances.} Cited 15 Jan 1999.
%
% Journal article
\bibitem{psysoc-journal} Harris, M., Karper, E., Stacks, G., Hoffman, D., DeNiro, R., Cruz, P., et al. (2001). Writing labs and the Hollywood connection. \textit{J Film} Writing, 44(3), 213--245.
%
% Contribution 
\bibitem{psysoc-contrib} O'Neil, J.~M., \& Egan, J. (1992). Men's and women's gender role journeys: Metaphor for healing, transition, and transformation. In B.~R. Wainrig (Ed.), \textit{Gender issues across the life cycle} (pp. 107--123). New York: Springer.
%
% Journal article by DOI
\bibitem{psysoc-DOI}Kreger, M., Brindis, C.D., Manuel, D.M., Sassoubre, L. (2007). Lessons learned in systems change initiatives: benchmarks and indicators. \textit{American Journal of Community Psychology}, doi: 10.1007/s10464-007-9108-14.
%
%
% Use the following syntax and markup for your references if 
% the subject of your book is from the field 
% "Humanities, Linguistics, Philosophy"
%
\bigskip
%
% Journal article
\bibitem{humlinphil-journal} Alber John, Daniel C. O'Connell, and Sabine Kowal. 2002. Personal perspective in TV interviews. \textit{Pragmatics} 12:257--271
%
% Contribution 
\bibitem{humlinphil-contrib} Cameron, Deborah. 1997. Theoretical debates in feminist linguistics: Questions of sex and gender. In \textit{Gender and discourse}, ed. Ruth Wodak, 99--119. London: Sage Publications.
%
% Monograph
\bibitem{humlinphil-mono} Cameron, Deborah. 1985. \textit{Feminism and linguistic theory.} New York: St. Martin's Press.
%
% Online Document
\bibitem{humlinphil-online} Dod, Jake. 1999. Effective substances. In: The dictionary of substances and their effects. Royal Society of Chemistry. Available via DIALOG. \\
http://www.rsc.org/dose/title of subordinate document. Cited 15 Jan 1999
%
% Journal article by DOI
\bibitem{humlinphil-DOI} Suleiman, Camelia, Daniel C. O'Connell, and Sabine Kowal. 2002. `If you and I, if we, in this later day, lose that sacred fire...': Perspective in political interviews. \textit{Journal of Psycholinguistic Research}. doi: 10.1023/A:1015592129296.
%
%
%
\bigskip
%
%
% Use the following syntax and markup for your references if 
% the subject of your book is from the field 
% "Computer Science, Economics, Engineering, Geosciences, Life Sciences"
%
%
% Contribution 
\bibitem{basic-contrib} Brown B, Aaron M (2001) The politics of nature. In: Smith J (ed) The rise of modern genomics, 3rd edn. Wiley, New York 
%
% Online Document
\bibitem{basic-online} Dod J (1999) Effective Substances. In: The dictionary of substances and their effects. Royal Society of Chemistry. Available via DIALOG. \\
\url{http://www.rsc.org/dose/title of subordinate document. Cited 15 Jan 1999}
%
% Journal article by DOI
\bibitem{basic-DOI} Slifka MK, Whitton JL (2000) Clinical implications of dysregulated cytokine production. J Mol Med, doi: 10.1007/s001090000086
%
% Journal article
\bibitem{basic-journal} Smith J, Jones M Jr, Houghton L et al (1999) Future of health insurance. N Engl J Med 965:325--329
%
% Monograph
\bibitem{basic-mono} South J, Blass B (2001) The future of modern genomics. Blackwell, London 
%
\end{thebibliography}

\end{document}
